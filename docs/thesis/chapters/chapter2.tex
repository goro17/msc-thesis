\chapter{Overview on Conflict-free Replicated Data Types}

This chapter serves as a technical background on CRDTs, providing an overview of the concepts
and terminology used throughout the rest of the thesis.

\section{Fundamentals of CRDTs}
At their core, CRDTs are abstract data types that are designed to be replicated across multiple nodes,
and that exhibit the following properties:
\begin{itemize}
    \item any replica can be modified without coordinating with any other replicas;
    \item when any two replicas receive the same set of updates, they reach the same state.
\end{itemize}
This is achieved by adopting mathematically sound rules to ensure state convergence \cite{preguiça2018conflict}.

CRDTs ovecome the limitations of syncrhonous concurrency control by adopting a more decentralized approach.
A node in the network may apply a set of updates to its local state, before transmitting the same set to all
the other nodes; given that the network may be unreliable, the other nodes may receive the updates out of order,
which may lead to diverging states. CRDTs ensure that the replicas eventually converge to a consistent state by
defining robust conflict resolution (or merging) mechanisms, regardless of the order in which the updates are
received. Depending the particular data structure that is being replicated, properties such as commutativity,
associativity, and idempotency are enforced to guarantee state convergence.

\subsection{Strong Eventual Consistency}
CRDTs achieve a model of consistency known as Strong Eventual Consistency \cite{shapiro2011conflict}, in
which two replicas reach the same final state if they have received the same set of updates.

\paragraph{Eventual Consistency (EC)} An object is said to be \textit{Eventually Consistent} if it satisfies the
following properties:
\begin{itemize}
    \item \textbf{Eventual delivery:} An update delivered to a replica is eventually delivered to all replicas; 
    \item \textbf{Convergence:} Replicas that have delivered the same updates eventually converge to an equivalent state; 
    \item \textbf{Termination:} All method executions terminate. 
\end{itemize}

\paragraph{Strong Eventual Consistency (SEC)} An object is said to be \textit{Strongly Eventually Consistent} if it
is Eventually Consistent if it satisfies the following additional property:
\begin{itemize}
    \item \textbf{Strong Convergence:} Replicas that have delivered the same updates have equivalent state.
\end{itemize}

\section{Types of CRDTs}

\subsection{State-based CRDTs}

\subsection{Operation-based CRDTs}

\subsection{Delta-state CRDTs}