\chapter{
    Foundational Concepts on Conflict-free Replicated Data Types and Local-first Software
}

In this chapter, we will focus on laying the foundations on CRDTs, and explore their
relevance when applied to local-first software, and in general in decentralized systems. We
will start by investigating the emergence of CRDTs as a means to achieve data consistency in
highly available systems; we will then delve into the theoretical foundations of CRDTs. We
will conclude the chapter by exploring how the capabilities of CRDTs can be leveraged in
local-first software.

\section{The Challenge of Data Consistency in Distributed Systems}

The proliferation of distributed computing paradigms, driven by the need for systems that are
highly available and performant, has brought the challenge of data consistency to the
forefront. As applications increasingly span multiple devices and geographical locations,
ensuring that all users and systems operate on a coherent view ot shared data becomes a
complex endeavor.

Replication approaches are frequently employed to address this challenge. By providing
multiple copies of data, called \textit{replicas}, across several nodes, the system can
ensure that users may access the same piece of information from multiple locations,
regardless of any potential system failure that may disrupt communication between the nodes
of the network. Within this effort, however, arises the fundamental problem of making sure
that the data remains consistent across all replicas, especially in situations where
multiple users may concurrently update such data. Moreover, achieving a strongly consistent
system comes at the cost of latency, as it requires replicas to adopt robust coordination
mechanisms, such as distributed consensus algorithms or two-phase commits, before finalizing
an update operation. This trade-off has been formalized in literature by the CAP Theorem,
which identifies the most desirable properties of a distributed system as
\textit{consistency}, \textit{availability}, and \textit{partition tolerance}, and
states that a real-world system can only simultaneously provide two out of these three
properties \cite{gilbert2002brewers}.

In many modern applications, particularly those operating over unreliable networks or
prioritizing user experience, availability and partition tolerance remain crucial, leading
to the need for weaker consistency models capable of functioning effectively under these
constraints. One such model, known as \textit{optimistic replication}, addresses concurrency
control among replicas by letting the data be accessed "optimistically" (without a priori
synchronization), propagating the resulting updates in the background, and fixing conflicts
after they happen \cite{saito2005optimistic}. Another prominent example of weak consistency
model, named \text{eventual consistency} (EC), allows for replicas to temporarily diverge,
but guarantees that, when updates to the data are no longer issued, they will converge to
an identical state \cite{terry1994session}. This model significantly enhances availability
and responsiveness, as operations can proceed without waiting for acknowledgement from other
peers. However, if concurrent updates are not reconciled in a principled manner, replicas may
end up in an inconsistent state, leading to erroneous behavior within the system. An
alternative approach to conflict resolution has led to the emergence of CRDTs, which provide
a mathematically sound framework for achieving state convergence without the need for
synchronization among replicas \cite{preguiça2018conflict}.

\section{Foundational Concepts on Conflict-free Replicated Data Types}

Conflict-free Replicated Data Types (CRDTs) are abstract data types specifically engineered
for replication across multiple processes in a distributed system. A defining characteristic
of CRDTs is that any replica can be modified without requiring immediate coordination with
other replicas \cite{shapiro2011conflict}. This property is fundamental to achieving
high availability and low latency in distributed applications.

\subsection{Strong Eventual Consistency}
CRDTs are designed to guarantee a strong form of eventual consistency known as Strong
Eventual Consistency (SEC)\cite{shapiro2011conflict}, in which two replicas reach the same
final state if they have received the same set of updates.

\paragraph{Eventual Consistency (EC)} An object is said to be \textit{Eventually Consistent}
if it satisfies the following properties:
\begin{itemize}
    \item \textbf{Eventual delivery:} An update delivered to a replica is eventually delivered
        to all replicas; 
    \item \textbf{Convergence:} Replicas that have delivered the same updates eventually
        converge to an equivalent state; 
    \item \textbf{Termination:} All method executions terminate. 
\end{itemize}

\paragraph{Strong Eventual Consistency (SEC)} An object is said to be
\textit{Strongly Eventually Consistent} if it is Eventually Consistent if it satisfies the
following additional property:
\begin{itemize}
    \item \textbf{Strong Convergence:} Replicas that have delivered the same updates
        converge to an equivalent state.
\end{itemize}

SEC achieves state convergence without requiring complex, ad-hoc conflict resolution logic
to be implemented by the developer. Instead, the "conflict-free" nature is embedded within
the design of the data type itself; concurrent operations are either inherently commutative
or are resolved by a deterministic merge procedure built into the CRDT.

\section{State-based vs. Operation-based CRDTs}

The design and implementation of CRDTs primarilys follow two distinct strategies for
replication: \textit{state-based} and \textit{operation-based}
\cite{shapiro2011comprehensive}.

\subsection{State-based CRDTs (CvRDTs)}

State-based CRDTs, also known as convergent replicated data types (CvRDTs), operate on the
principle of replicas exchanging their entire current state. When an update to a replica's
state is issued, the replica first applies the update locally, and then broadcasts the
resulting state to the other replicas. The replicas receiving the updated state employ a
\textit{merge} function to combine the received state with their own local state. for
convergence to be guaranteed, this merge function must possess three key mathematical
properties:

\begin{itemize}
    \item \textbf{Commutativity:} the order in which states are merged does not affect the
        outcome;\\
        \begin{equation}
            \begin{aligned} x \cdot y = y \cdot x \end{aligned} 
        \end{equation}
    \item \textbf{Associativity:} the order in which merges are grouped does not affect the
        outcome;\\
        \begin{equation}
            \begin{aligned} (x \cdot y) \cdot z = x \cdot (y \cdot z) \end{aligned}
        \end{equation}
    \item \textbf{Idempotency:} merging identical states produces the same resulting state
        (even when the merge is repeated multiple times).
        \begin{equation}
            \begin{aligned} x \cdot x = x \end{aligned}
        \end{equation}
\end{itemize}

Thanks to these properties, state-based CRDTs can achieve state convergence among replicas,
as long as the replicas have received the same set of updates.

\subsection{Operation-based CRDTs (CmRDTs)}
Operation-based CRDTs, also known as commutative replicated data types (CmRDTs), take a
different approach. Instead of transmitting the entire updated state, replicas broadcast
the update \textit{operations} themselves as they occur locally. These operations are then
applied locally by the receiving replicas. For CmRDTs to ensure convergence, the operations
must be designed to be commutative when applied concurrently. Furthermore, CmRDTs typically
impose stricter constraints on the underlying communication infrastructure. Guarantees such
as exactly-once delivery and causal delivery of operations are often necessary to ensure
that all operations are applied consistently across replicas.

\subsection{Delta-state CRDTs}
A significant optimization, particularly for state-based CRDTs, is the concept of
delta-state CRDTs (\textit{$\delta$-CRDTs}). Instead of transmitting the entire state,
$\delta$-CRDTs only need to transmit the incremental changes (or \textit{deltas}) that have
occurred since the last state update. These deltas are then merged at the receiving
replicas. This approach overcomes the limitations of state-based replication, which can
often be computationally expensive if the state object to transmit is large in size.

\section{Application of CRDTs in Local-first Software}
Advancements in modern distributed systems have led to the development of applications that
function predominantly "on the cloud". These types of applications, such as Google Docs or
Trello, usually require a strong dependency on a centralized authority (usually assumed by
one or more remote servers) to handle concurrent data transformations. Users often benefit
from these applications for their capability to store data --- documents, presentations,
spreadsheets, etc. --- that can then be accessed from multiple devices, and often sharing
this same data with other users. As the user's request to access their data on the cloud
passes through remote servers, one could question the degree of ownership assumed by the
user over their data, as such ownership is oftentimes "shared" with the cloud provider who
is hosting the same data\footnote{
    A well-known colloquial phrase, often cited in popular culture to demistify cloud
    computing, states that "There is no cloud, it's just someone else's computer."
    \url{https://www.chriswatterston.com/article/success-of-my-there-is-no-cloud-sticker}
}. In addition, cloud-centric applications introduce several drawbacks from the user's
standpoint. For instance, applications may become unusable offline, and performance could be
dictated by network latency.

A new trend in software development has emerged in recent years and has posed itself in
direct contrast to the traditional cloud-based architectures. \textit{Local-first software}
favor a more user-centric approach by prioritizing storing and processing data directly on
the user's device, with remote servers typically relegated to roles such as syncrhonization
facilitators or backup providers \cite{kleppmann2019local}. Aside from this approach
resulting in immediate access to the data and lower latency, local-first applications allow
users to have complete control over their data, and to not depend on network connectivity to
operate the application itself.

Due to their decentralized nature, CRDTs are perfectly suitable to enhance the
functionalities of local-first applications --- especially when it comes to data that is
shared among multiple users. 