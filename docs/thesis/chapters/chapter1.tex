\chapter{Introduction}

The modern computing landscape has moved towards building distributed systems as a means
to fulfill the necessity of delivering high volumes of data to a continuously growing
number of users over the Internet. It has become commonplace to have systems span over a
network (cluster) of machines -- or \textit{nodes} -- which communicate with each other
in a coordinated fashion. Compared to monolithic systems that assign the entire workload
to a single machine, this approach efficiently addresses three fundamental properties
that the majority of modern systems seeks to achieve.

\begin{itemize}
	\item \textit{Scalability}: the system's ability to adapt to the scale of the
		workload by either allocating a larger quantity of resources on one node --
		\textit{vertical scaling} -- or dividing it among different nodes --
		\textit{horizontal scaling};
	\item \textit{Resiliency}: the system's ability to tolerate eventual faults that may
		affect the correct operation of one or more nodes;
	\item \textit{Efficiency}: the system's responsiveness expressed in terms of time and space
		\footnote{An ideal system operates under low latency and high throughput}.
\end{itemize}

Advancements in this field sought to optimally achieve these pivotal demands have led to
the emergence of more advanced paradigms, such as \textit{cloud computing}, which poses
itself as a model that enables ubiquitous, on-demand access to a shared pool of
configurable computing resources -- networks, servers, storage, services -- provisioned
with minimal management effort\cite{hamdaqa2012cc}. The advent of cloud computing has
fueled an increasing interest in applications that are built to empower collaborative
tasks. Platforms like Google Drive\footnote{https://drive.google.com}, Microsoft
OneDrive\footnote{https://onedrive.live.com}, and
Dropbox\footnote{https://dropbox.com} enable groups of two or more users to engage
in common tasks by sharing access to a set of resources -- documents, spreadsheets,
presentations, images, ... -- from anywhere in the world, in real
time\cite{ellis1989groupware}.

Within collaborative applications lies the challenge of making sure that concurrent
updates are applied seamlessly and without conflicts, especially in situations where
users may be working across different geographical locations with varying network
latencies. Despite the risk of network connectivity being intermittent, users expect
their collaborative experience to be responsive, their changes to be preserved, and
conflicts to be resolved seamlessly without loss of data. Traditional approaches to
this challenge, like the one adopted by Google Drive, heavily rely on centralized
conflict resolution strategies such as \textit{Operational Transformation} (OT), where a
single node acts as the authoritative source of truth and mediates all concurrent
changes\cite{sun1998ot}. While this mechanism has proven to be reliable, a compelling
alternative could be achieved by employing a truly decentralized, peer-to-peer (P2P)
network, over which data is replicated across multiple nodes and concurrent changes are
handled without the need of a centralized authority.

Conflict-free Replicated Data Types (CRDTs) offer a powerful solution suitable to support
highly available file sharing and collaborative systems. CRDTs are data structures that
allow the system to update any replica independently and concurrently, without
coordinating with other replicas\cite{almeida2024crdts}. The core benefit is that the
data type itself contains an algorithm that automatically resolves any inconsistencies,
guaranteeing that replicas will eventually converge to an identical state, even in the
case in which a node may receive updates out of order.

The formalized problem addressed by this thesis is the design and implementation of a
Proof-of-Concept (PoC) replicated file storage solution that leverages CRDTs to
synchronize file contents and metadata. The system is conceptualized as a P2P network
where each participating node maintains a local replica of the storage system. All
changes made on any node are propagated to other nodes and merged automatically using
CRDT algorithms. For the PoC, the complexities of a true P2P network are abstracted away
by using a relay server. This server acts as a message broker, broadcasting updates from
one node to all other connected nodes, thereby simulating the propagation of changes in a
P2P environment. The core of the system is built using the
\textit{PyCRDT library}\footnote{https://github.com/y-crdt/pycrdt}, a Python
binding for the popular \textit{Yjs}\footnote{https://yjs.dev/} CRDT
framework. This allows for the use of mature and well-tested CRDT implementations to
model the file system's directory structure, file contents, and metadata. The system
is complemented by a simple web-based client that provides a user interface for
seamless file management, where users can upload their files and enforce a robust data
integrity guarantee by attaching a synthetic digital signature to each file. Other users
can validate such signatures through an offline-first validation mechanism, using the
data that was received through CRDTs.

The rest of this thesis is structured as follows:
\begin{itemize}
	\item In \textbf{Chapter 2}, we provide a comprehensive overview of the foundational
		concepts of CRDTs, along with their taxonomy, core data structures, and conflict
		resolution strategies.
	\item In \textbf{Chapter 3}, we analyze existing CRDT implementations and frameworks,
		with a particular focus on the Yjs ecosystem, which forms the basis of the PoC.
	\item In \textbf{Chapter 4}, we delve into the design and implementation process a
		functioning PoC that demonstrates the feasibility of using CRDTs for file
		synchronization in a simulated P2P environment.
	\item In \textbf{Chapter 5}, we evaluate the PoC's performance, identify its
		strengths and limitations, and propose directions for future work.
\end{itemize}
