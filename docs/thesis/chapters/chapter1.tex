\chapter{Introduction}

Modern distributed systems often leverage the potential of serving replicated data at multiple geographic
locations to guarantee low latency and high availability. A highly desirable property in systems tailored for this
purpose is that the data should remain consistent across replicas; such a constraint dictates that the replicas
must eventually converge to the same state. However, achieving consistency in highly available distributed systems
poses a significant challenge, particularly when network outages may occur or when systems must operate without
centralized coordination. 

There are several methods for achieving consistency in highly available systems. One prominent approach is the use
of Conflict-free Replicated Data Types (CRDTs). CRDTs are specialized data structures that can be replicated across
multiple nodes, and are capable of ensuring that the replicas eventually converge to a consistent state without
requiring explicit coordination from a central authority. Their decentralized nature enabled each node of the
network to operate independently from others, making CRDTs particularly well-suited for scenarios where network
connectivity is unreliable.

Given their nature, CRDTs have been implemented in a wide range of decentralized applications, such as
collaborative editing tools and distributed databases. Local-first software --- an approach to application
development that prioritizes local data storage and processing --- represents another field where CRDTs have shown
particular effectiveness, as they enable applications to operate seamlessly offline while synchronizing changes
over the network, sharing updates with other peers as they come.

This thesis investigates the practical implementation of CRDTs within local-first software architectures through the
development of a digital signature management application --- named \textit{CRDTSign}. The application leverages CRDTs
to manage a digital signature store that is shared among multiple nodes of a distributed network, and focuses on
maintaining consistency across the nodes. Built using the Python programming language, the application makes use of a
Python port of the popular \textit{Yjs}\footnote{\url{https://docs.yjs.dev/}} library to build custom CRDT-based data
structures. Through this case study, we will evaluate the effectiveness of CRDTs within the design of robust and
conflict-free distributed systems.

%--------------------------------------------------------------------------------------------------------------
% The thesis is organized as follows. Chapter 2 presents the background of the problem and the CRDTs used in 
% the application. Chapter 3 presents the implementation of the application, and chapter 4 presents the results
% of the experiments conducted. Chapter 5 presents the conclusion of the thesis.
%--------------------------------------------------------------------------------------------------------------