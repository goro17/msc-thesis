\chapter{Evaluation of the Proof-of-Concept Application and Conclusions}

\section{Summary of the Proof-of-Concept Application}

The aim of this thesis was to illustrate Conflict-free Replicated Data Types and their
utility in modern distributed software. This implied developing a Proof-of-Concept
application -- CRDTSign -- a replicated file storage solution that enables a group
of users to collaborate on a shared file storage. Each user has the possibility of
uploading a file -- and certifying ownership on it through the means of a cryptographic
signature -- and receiving files from other peers. Thanks to CRDTs, the PoC has the
capability to operate in a decentralized manner -- each operation can be individually
applied from each node, and the system's shared state is kept (eventually) consistent,
without the need of a central node that coordinates operations and the order in which
they are applied. This is an improvement over similar applications that leverage
synchronization technologies such as Operational Transformation (OT), which elects a
central node to manage operations for the rest of the network, with the goal of
resolving possible conflicts.

In the following sections, we will provide a preliminary evaluation on the PoC's behavior
with respect to \textit{scalability}, and then discuss some areas of improvement over the
currently developed solution.

\section{Evaluation of the Proof-of-Concept Application}

While the behavior of CRDTSign with respect to functional and tecnhnical requirements has
been established, its viability in a production-like environment depends on its
scalability profile. Consequently, this section aims to introduce a preliminary
evaluation of the solution that was performed with an experimental setup, and the
resulting metrics that were measured throughout extensive testing. Through this
analysis, we will discuss how the system behaves as demand scales, and assess the
throughput and latency requirements achieved by the proposed solution.

\subsection{Overview of the experimental setup for scalability testing}

In an attempt to simulate real-world use cases 

\section{Considerations for Future Work}
