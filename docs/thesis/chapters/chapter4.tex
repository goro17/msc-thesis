\chapter{Design and Implementation of the Proof-of-Concept Application}

Whereas the previous chapters were dedicated in providing a comprehensive literature
review on the topic, we will now focus on the practical implementations of integrating
solutions based on CRDTs in distributed software. In order to achieve this, we will
analyze a use case in which the implementation of CRDT-based data structures proves to be
\textit{valuable}, and document the development of a Proof-of-Concept application built
around that use case.
% TODO: replace the word valuable with something different 

\section{Defining the Use Case for the Proof-of-Concept}

The traditional infrastructure surrounding the broad family of collaborative applications
-- e.g., Google Docs, OneDrive, and many more -- have constantly relied on centralized
coordination paradigms, such as OT. While the approach of designating a central
coordinator for conflict resolution offers distinct advantages -- primarily revolving
around enabling near real-time, non-blocking collaboration and preserving the user's
intention -- there are some contexts in which centralization may present significant
drawbacks, mostly in terms of partition tolerance. For instance, the central coordinator
may be put in a position to be the single point of failure; if it is unreachable, there
is no other node that can approve or order changes, and the operativity of the entire
system collapses\footnote{Some countermeasures can be adopted, such as implementing a
leader election mechanism for designating a new central coordinator node among the nodes
of the network. Nevertheless, this approach would introduce a further layer of complexity
in the distributed system, and would require appropriate handling to enforce
convergence.}.

This precisely illustrates the fundamental advantage of decentralized, peer-to-peer (P2P)
architectures based on CRDTs; through replication across multiple nodes, information is
maintained with high availability and consistency -- i.e., compliant with SEC
constraints. Furthermore, if one or more nodes of the system fail, the system's
operativity is still guaranteed, as updates travel to all active nodes -- whereas the
faulty nodes will reconcile with the system's consistent state, once they come back
online. This approach eliminates the threat of a single point of failure that would
compromise system-wide reliability.

To illustrate this principle, we present a proof-of-concept implementation in the form of
a replicated file storage system that leverages CRDTs to achieve eventual consistency
without the need of centralized conflict resolution mechanisms. This system enables
multiple users to execute fundamental file operations -- including file sharing and
deletion -- on a shared storage infrastructure. 

Furthermore, the solution provides a \textit{file signature} mechanism, through which an
unique user provides a secure method of certifying its ownership over the same file, and
on-demand validity check over the signature can be requested by any other user.

In the following sections, we will refer to the proof-of-concept by the name of
\textit{CRDTSign}. 

\subsection{Proof-of-Concept Scope and Boundaries}

It is worth underlining that the solution developed through CRDTSign represents a toy
implementation, rather than production-grade software. Its primary objective is to
evaluate the capabilities of CRDTs in contemporary distributed peer-to-peer (P2P)
systems. As such, several high-level requirements must be satisfied:

\begin{itemize}
	\item consistent with the principles of decentralized applications, the backend
		logic is not centralized on a single node, but is served independently at each
		participating node;
	\item because CRDTSign aims to provide an example of modern collaborative software,
		convergence among nodes must be achieved in near real-time -- to satisfy end-user
		expectations for system responsiveness.
\end{itemize}

Conversely, given that the digital signature mechanism is integral to the solution --
serving as a certificate of ownership for files shared by users -- the system must adopt
a cryptographic signature scheme that ensures:

\begin{itemize}
	\item \textit{Integrity}: signatures cannot be altered without detection;
	\item \textit{Authenticity}: the signer's identity can be verified;
	\item \textit{Non-repudiation}: the signer cannot deny having signed the data.
\end{itemize}

As an additional requirement, the system adopts a data retention policy, which ensures
that files whose creation timestamp exceeds a configurable interval -- e.g., 90 days --
are automatically removed from the shared storage. This is made possible to prevent cases
in which the storage's memory becomes saturated over time by files which are unused or
obsolete.

While the implemented solution satisfies these core requirements, certain architectural
constraints limit its robustness. For instance, the P2P architecture was not constructed
from the ground up; instead, it was abstracted through the use of a relay server. Rather
than enabling direct node-to-node communication, all messages are routed through the
relay server, which in turn broadcasts them to all peers -- thereby simulating a P2P
environment. This design choice does not fundamentally compromise the decentralized
nature of the system, as concurrency management logic remains managed across individual
nodes, rather than being centralized within the relay server. Additionally, the CRDT data
structure was sourced from an open-source library, and was not custom-made for this
application. As we will discuss in the next chapter, a more robust and scalable solution
could be achieved by designing a domain-specific CRDT structure and message exchange
protocol tailored to the characteristics of the shared data.

\subsection{User Story}

We will now discuss several use cases of CRDTSign by describing the functionalities of
the system that are available to the end-user.

\paragraph{Uploading a file to storage}
User A

\section{Requirements Analysis}

\subsection{Defining the Correctness Criteria}

\section{System Architecture}

\section{Software Components}

\subsection{Backend}

\subsection{Frontend}

\subsection{Relay Server}

\section{Production Deployment}

\section{Development of the Complementary Mobile Application}
