\chapter{Design and Implementation of the Proof-of-Concept Application}

Whereas the previous chapters were dedicated in providing a comprehensive literature
review on the topic, we will now focus on the practical implementations of integrating
solutions based on CRDTs in distributed software. In order to achieve this, we will
analyze a use case in which the implementation of CRDT-based data structures proves to be
\textit{valuable}, and document the development of a Proof-of-Concept application built
around that use case.
% TODO: replace the word valuable with something different 

\section{Defining the Use Case for the Proof-of-Concept}

The traditional infrastructure surrounding the broad family of collaborative applications
-- e.g., Google Docs, OneDrive, and many more -- have constantly relied on centralized
coordination paradigms, such as OT. While the approach of designating a central
coordinator for conflict resolution offers distinct advantages -- primarily revolving
around enabling near real-time, non-blocking collaboration and preserving the user's
intention -- there are some contexts in which centralization may present significant
drawbacks, mostly in terms of partition tolerance. For instance, the central coordinator
may be put in a position to be the single point of failure; if it is unreachable, there
is no other node that can approve or order changes, and the operativity of the entire
system collapses\footnote{Some countermeasures can be adopted, such as implementing a
leader election mechanism for designating a new central coordinator node among the nodes
of the network. Nevertheless, this approach would introduce a further layer of complexity
in the distributed system, and would require appropriate handling to enforce
convergence.}.

This precisely illustrates the fundamental advantage of decentralized, peer-to-peer (P2P)
architectures based on CRDTs; through replication across multiple nodes, information is
maintained with high availability and consistency -- i.e., compliant with SEC
constraints. Furthermore, if one or more nodes of the system fail, the system's
operativity is still guaranteed, as updates travel to all active nodes -- whereas the
faulty nodes will reconcile with the system's consistent state, once they come back
online. This approach eliminates the threat of a single point of failure that would
compromise system-wide reliability.

To illustrate this principle, we present a proof-of-concept implementation in the form of
a replicated file storage system that leverages CRDTs to achieve eventual consistency
without the need of centralized conflict resolution mechanisms. This system enables
multiple users to execute fundamental file operations -- including file creation and
deletion -- on a shared storage infrastructure. In order to rigorously check the
	integrity of files that are shared over the neteork

\subsection{Proof-of-Concept Scope and Boundaries}

\subsection{User Story}

\section{Requirements Analysis}

\subsection{Defining the Correctness Criteria}

\section{System Architecture}

\section{Software Components}

\subsection{Backend}

\subsection{Frontend}

\subsection{Relay Server}

\section{Production Deployment}

\section{Development of the Complementary Mobile Application}
