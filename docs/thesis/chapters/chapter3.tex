\chapter{Principles of Local-First Software}

\section{Defining Local-First Software}
Advancements in modern distributed systems have led to the development of applications that
function predominantly "on the cloud". These types of applications, such as Google
Docs\footnote{\url{https://docs.google.com/}} or
Trello\footnote{\url{https://trello.com/it}}, usually require a strong dependency on a
centralized authority (usually assumed by one or more remote servers) to handle concurrent
data transformations. Users often benefit from these applications for their capability to
store data --- documents, presentations, spreadsheets, etc. --- that can then be accessed
from multiple devices, and often sharing this same data with other users.

As the user's request to access their data on the cloud passes through remote servers, one
could question the degree of ownership assumed by the user over their data, as such
ownership is oftentimes "shared" with the cloud provider who is hosting the same
data\footnote{
    A well-known colloquial phrase, often cited in popular culture to demistify cloud
    computing, states that "There is no cloud, it's just someone else's computer."
    \url{https://www.chriswatterston.com/article/success-of-my-there-is-no-cloud-sticker}
}. In addition, cloud-centric applications introduce several drawbacks from the user's
standpoint. For instance, applications may become unusable offline, and performance could be
dictated by network latency.

An ongoing trend in software development has emerged in recent years and has posed itself in
direct contrast to the traditional cloud-based architectures. A novel set of principles,
which falls under the name of \textit{"local-first software"}, favors a more user-centric
approach to software design by prioritizing storing and processing data directly on
the user's device, with remote servers typically relegated to roles such as syncrhonization
facilitators or backup providers \cite{kleppmann2019local}. Aside from this approach
resulting in immediate access to the data and lower latency, local-first applications allow
users to have complete control over their data, and to not depend on network connectivity to
operate the application itself.

\subsection{Key Benefits of Local-First Software}
\begin{itemize}
    \item \textbf{Offline Access}
    \item \textbf{Lower Latency}
    \item \textbf{Complete Ownership}
    \item \textbf{Ubiquitous Access}
    \item \textbf{Resilience}
\end{itemize}
